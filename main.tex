% TODO
% [  ]  ä,ö,ü

\documentclass[11pt,a4paper]{article}
% \usepackage[ngerman]{babel}							% enables Hyphenation for german
% \usepackage[babel, german=quotes]{csquotes}
\usepackage{subfigure}								% enables subfigures
\usepackage{amsmath}								% enhances math output
\usepackage{amsfonts}								% additional math fonts
\usepackage{graphicx}								% alternativ graphic interface
\usepackage{color,listings}	 						% codesequences 
\usepackage{import}
\usepackage{cite}
% \usepackage{biblatex}

\begin{document}

\section{Einführung} \label{sec:introduction}
Die populärsten Blockchain basierten Kryptowährungen wie Bitcoin und Ethereum haben noch einige technische Mängel.
Insbesodere der Energieverschwenderischen Proof-Of-Work kann auch wegen den den grossen 
und damit mächtigen Mining Pools zum Problem werden.
Zudem stossen Bitcoin und in absehbarer Zeit auch Ethereum an ihre Kapazitätsgrenzen (Transaktionen pro Zeit).

Einige Leute um den MIT Professor und Turing Award Winner Silvio Micali haben das alternative Verfahren Algorand vorgestellt.
Dieses kommt ohne Proof-of-Work aus und soll eine deutlich grössere Kapazität (mehr Transaktionen in der selben Zeit) bereit stellen
als die meisten anderen bekannten Blockchain basierte Lösungen.

Soweit wir das beurteilen könne, ist dieses System mit einigen zentausenden Nodes getestet worden,
wird aber zur Zeit noch nicht produktiv eingesetzt.
Zudem sind die zur Verfügung stehenden Quellen bezüglich Algorand grössten Teils auf eine kleine Gruppe zurück zu führen,
die meisten Arbeiten im CSAIL am MIT. % Computer Science and Artificial Inteligence Laboratory

Transaction confirmation 1 minute \cite[1. Introduction]{Gilad:2017:ASB:3132747.3132757}



\cite{Gilad:2017:ASB:3132747.3132757}
\cite{Chen:2017}
\cite{ScalingConsensus}





\newpage

% \begin{figure}[htb]
% 	\centering
% 	\includegraphics[width=\textwidth, angle=0, clip, trim=8mm 100mm 90mm 8mm]{bilder/pksys_blockschaltbild}
% 	\caption{Blockschaltbild des drahtlosen Patientenklingelsystem}
% 	\label{fig:pksys_blockschaltbild}
% \end{figure}

% \bibliographystyle{plain}
\bibliographystyle{apalike} % makes reference labels like [Redmond et al., 2016]
\bibliography{algorand}{}

\end{document}
