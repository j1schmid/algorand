\documentclass[11pt,a4paper]{article}
\usepackage[ngerman]{babel}							% enables Hyphenation for german
% \usepackage[babel, german=quotes]{csquotes}
\usepackage{subfigure}								% enables subfigures
\usepackage{amsmath}								% enhances math output
\usepackage{amsfonts}								% additional math fonts
\usepackage{graphicx}								% alternativ graphic interface
\usepackage{color,listings}	 						% codesequences 
\usepackage{import}
\usepackage{cite}
\usepackage{enumitem}
% \usepackage{biblatex}
\usepackage{fancyhdr}			%% fuer Pagestyle fancyhdr mit eigenen Kopf und Fusszeilen


\author{Heinz Hofmann und Jonas Schmid}

%% Pagestyle mit eigenen Kopf und Fusszeilen
\pagestyle{fancy}
\fancyhf{}								%% leerraeumen
% \fancyhead[L]{\includegraphics[height = 20pt]{logo.png}}
% \fancyhead[R]{Simulationstools}
\renewcommand{\headrulewidth}{0.0pt}	%% obere Trennlinie
% \fancyfoot[C]{\thepage}					%% Seitennummer
\fancyfoot[L]{Heinz Hofmann und Jonas Schmid}	
\fancyfoot[R]{20. Dezember 2017}		
\renewcommand{\footrulewidth}{0.4pt}	%% untere Trennlinie


\begin{document}

\section*{ \center \textbf{\Large Algorand: Eine vielversprechende Blockchain basierte Kryptow\"ahrungen}}
% \section{Einf\"uhrung} \label{sec:introduction}

Die popul\"arsten Blockchain basierten Kryptow\"ahrungen wie Bitcoin und Ethereum haben noch einige technische M\"angel.
Insbesodere der Energieverschwenderischen Proof-Of-Work kann auch wegen den grossen 
und damit m\"achtigen Mining Pools zum Problem werden.
Zudem stossen Bitcoin und in absehbarer Zeit auch Ethereum an ihre Kapazit\"atsgrenzen (Transaktionen pro Zeit).
Einige Leute um den MIT Professor und Turing Award Winner Silvio Micali haben das alternative Verfahren Algorand vorgestellt.
Dieses verwendet eine Art Proof-of-Stake mit einem speziel angepassten Byzantine Agreement Verfahren und soll eine deutlich gr\"ossere Kapazit\"at bereit stellen als bisherige Blockchain basierte L\"osungen.
Die "Transaction Confirmation Time" betr\"agt weniger als eine Minute und es k\"onnen rund 125 mal mehr Transaktionen get\"atigt werden als bei Bitcoin \cite[Introduction]{Gilad:2017:ASB:3132747.3132757}.
Soweit wir das beurteilen k\"onnen, wird das System aktuell noch nicht produktiv eingesetzt.
Es wurde jedoch ein Prototyp mit einigen hunderttausend Nodes erfolgreich simuliert \cite[Kapitel Implementation \& Evaluation]{Gilad:2017:ASB:3132747.3132757}.
Die zur Verf\"ugung stehenden Quellen sind gr\"ossten Teils auf eine kleine Gruppe zur\"uck zu f\"uhren,
die meisten Arbeiten im CSAIL am MIT. % Computer Science and Artificial Inteligence Laboratory
Nachfolgend beziehen wir uns mehrheitlich auf das Paper \cite{Gilad:2017:ASB:3132747.3132757}, welches Algorand resp. den Algorithmus detailiert und vollst\"andig beschreibt, ein grober \"Uberblick wird im Abschnitt Overview oder im Betrag \cite{ScalingConsensus} gegeben.

Algorand basiert auf einer Blockchain sehr \"ahnlich wie Bitcoin.
Wie bereits erw\"ahnt wird eine Art Proof-of-Stake Verfahren angewendet.
Dabei wird jeder neue Block durch ein Committee bestehend aus weniger als 100 Nodes\footnote{Bei der Testimplementation wurden 26 Nodes angestrebt.} mittels Byzantine Agreement zusammengestellt resp. ausgewählt.
Das Committee wird zufällig aber gewichtet mit den jeweiligen Balances der Nodes für jeden neuen Block und auch für jeden Schritt im Byzantine Agreement Verfahren neu ausgew\"ahlt.
Das heisst ein Node welcher viel "Geld" im System hat, bekommt \"ofter ein Mandat im Committee als ein Node welcher wenig Geld im System hat \footnote{Es k\"onnen auch mehrere Committee Mandate auf einen einzelnen Node fallen.}, damit ist Algorand nicht anf\"allig auf \textit{Sybill attacks}.
Das Auswahlverfahren - die sogenannte \textit{cryptographic sortition} - kann jeder Node für sich selbst durchführen, dies kann vereinfacht als Pseudozufallszahlengenerator betrachtet werden.
Dieser wird mit einem \textit{seed} abhängig vom vorangehenden Block resp. der vorangehenden Byzantine Agreement Runde und dem \textit{private key} initialisiert.
Abh\"angig von dieser Zufallszahl ist dann bestimmt ob der jeweilige Node als \textit{committee} Mitglied fungiert.
Der Zufallszahlengenerator (\textit{verifiable random function} kurz VRF) ist so konstruiert,
dass die Ausgegebene Zufallszahl und damit die Rollenzuteilung von jedem Node \"uberpr\"uft werden kann,
nat\"urlich aber ohne den \textit{private key} des jeweiligen Nodes zu kennen.
Die jeweiligen \textit{committee} Mitglieder setzten dann einen Block aus den im bekannten \textit{unconfirmed transations} zusammen und versendet diesen dann ins gesammte Netztwerk.
Ausgewählt wird dann der Block des Mitgliedes mit der höchsten Priorität.
Falls nun ein b\"osartiger Node in dieser Position verschiedene Bl\"ocke oder einen fehlerhaften Block
verteilt, kann sich das \textit{commitee} nicht einigen und erweitert die Blockchain um einen leeren Block ohne Transaktionen.

TODO da ist das ganze für mich noch leicht diffus.

Dann wird mit einem neu bestimmten \textit{comittee} das ganze wiederhohlt.

\newpage
\paragraph*{Block zur Blockchain hinzuf\"ugen }
Das Hinzuf\"ugen eines Blocks zur Blockchain vollzieht sich in drei Schritten.
\begin{enumerate}[label=\arabic*)]
	\item \textbf{Block erzeugen und verteilen}\\
	\textbf{Jeder} Node f\"uhrt bei sich die Sortition durch, um herauszufinden, ob er einen Block verteilen darf. Falls ja f\"ugt er alle Transaktionen, die er m\"ochte dem Block hinzu und "broadcasted" diesen dem Netzwerk. Er wird einer von maximal ca. 70 Nodes sein, der einen Block broadcastet. Jeder dieser max. 70 broadcasteten Blöcke hat aber eine gewisse Priorität, welche ebenfalls aus der Sortition kommt. 
	\item \textbf{Byzantine Agreement}\\
	Nun wird in zwei Schritten das Problem "einen aus vielen Blöcken auswählen" auf das Problem "entweder den proposten Block oder einen leeren Block auszuwählen"
	\begin{enumerate}[label=\Roman*)]
		\item Nach einer bestimmten Wartezeit(zur Synchronisation) gibt ein Gremium von zufällig durch Sortition ausgewählten Nodes eine Stimme für je einen Block ab. Sie wählen jeweils von den für sie sichtbaren Blocks denjenigen mit der höchsten Priorität. Diese Stimme broadcasten Sie wieder durch das Netzwerk.
		\item Ein erneut durch Sortition zufällig gewähltes Gremium wartet bis entweder genügend Votes für einen Block zusammenkamen, oder bis eine bestimmte Zeit abgelaufen ist. 
		Sind genügend Votes zusammengekommen, wird der entsprechend gewählte Block promotet.
		Ist die Zeit abgelaufen, wird ein leerer Block promoted. 
		Alle gutartigen Nodes werden hier entweder denselben vollen Block oder einen leeren Block promoten. 
	\end{enumerate}
	\item \textbf{Binary Byzantine Agreement}\\
	Ab hier wird in einem Binären Byzantinischen Verfahren entweder der eine "volle" Block oder der leere Block gewählt. Falls es keine Fehlbaren Nodes gibt ist dieser Prozess in wieder zwei Schritten beendet. Wird der Prozess aber von Fehlbaren Nodes gezielt mit z.B. promotions von falschen Blöcken gestört(Was schon schwierig ist, weil diese Blöcke ins Gremium hätten gewählt werden müssen) kann dieser Vorgang bis zu 9 Schritte andauern.
	In jedem Schritt wird jeweils ein neues Gremium gewählt, welches aufgrund von der Vergangenheit Votes für den einen vollen oder einen leeren Block abgeben. Jeder dritte dieser Schritte wird ausserdem eine "gemeinsame" Münze geworfen, welche hilft Fortschritt zu garantieren. 
\end{enumerate}
Dieser ganze Prozess dauert ca. 20 Sekunden. Konservativerweise wurden die Wartezeiten der Nodes, bis sie voten so gewählt, dass dieser ganze Prozess rund eine Minute dauert.



\newpage

\paragraph*{Empfehlung f\"ur eine Vorlesung oder Seminar}
Das Paper \cite{Gilad:2017:ASB:3132747.3132757} w\"urde sich gut eignen f\"ur eine weitere "Buchdiskussion".
Es werden einige Aspekte bez\"uglich Distributed Systems/Ledger aufgegriffen welche 
im Buch "Distributed Ledger Systems" von Roger Wattenhofer tematisiert wurden, z.B. Byzantine Agreement.

\cite{Gilad:2017:ASB:3132747.3132757}
\cite{Chen:2017}


\newpage

% \begin{figure}[htb]
% 	\centering
% 	\includegraphics[width=\textwidth, angle=0, clip, trim=8mm 100mm 90mm 8mm]{bilder/pksys_blockschaltbild}
% 	\caption{Blockschaltbild des drahtlosen Patientenklingelsystem}
% 	\label{fig:pksys_blockschaltbild}
% \end{figure}

% \bibliographystyle{plain}
\bibliographystyle{apalike} % makes reference labels like [Redmond et al., 2016]
\bibliography{algorand}{}

\end{document}
